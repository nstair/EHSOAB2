%\documentclass[10pt]{article}

\documentclass[aip,jcp,amsmath,amssymb, reprint]{revtex4-1}
%\documentclass[aip,jcp,amsmath,amssymb, linenumbers, reprint]{revtex4-1}
%\documentclass[aip,jcp,amsmath,amssymb, preprint]{revtex4-1}

\usepackage{graphicx}
\usepackage{dcolumn}
\usepackage{amsfonts}
\usepackage{braket}
\usepackage{multirow}
\usepackage{threeparttable}
\usepackage{xspace}
\usepackage{verbatim}
\usepackage{mhchem}
\usepackage{soul}
\usepackage{ulem}
\usepackage{siunitx}
\usepackage{lineno}% Enable numbering of text and display math
%\linenumbers\relax % Commence numbering lines
\usepackage[colorlinks = true,
            linkcolor = blue,
            urlcolor  = black,
            citecolor = blue,
            anchorcolor = black]{hyperref}

\usepackage{amsmath}
\usepackage{mathtools}
\usepackage{xcolor}
\usepackage{xspace}
\usepackage{ifthen}

\newcommand*{\Eh}{$E_{\rm h}$\xspace}
\newcommand*{\Ecorr}{$E_{\rm{corr}}$\xspace}
\newcommand*{\RCMnorm}{$\norm{\pmb{\lambda}_{2}}^{2}_\mathrm{F}$\xspace}
\newcommand*{\dRCMnorm}{$\norm{\pmb{\delta\lambda}_{2}}^{2}_\mathrm{F}$\xspace}
\newcommand*{\nfci}{N_\mathcal{H}}
\newcommand*{\ncomp}{\mathcal{V}_X}

\providecommand{\norm}[1]{\lVert#1\rVert}

\setlength\linenumbersep{6pt}

\newcommand*\patchAmsMathEnvironmentForLineno[1]{%
  \expandafter\let\csname old#1\expandafter\endcsname\csname #1\endcsname
  \expandafter\let\csname oldend#1\expandafter\endcsname\csname end#1\endcsname
  \renewenvironment{#1}%
     {\linenomath\csname old#1\endcsname}%
     {\csname oldend#1\endcsname\endlinenomath}}% 
\newcommand*\patchBothAmsMathEnvironmentsForLineno[1]{%
  \patchAmsMathEnvironmentForLineno{#1}%
  \patchAmsMathEnvironmentForLineno{#1*}}%
\AtBeginDocument{%
\patchBothAmsMathEnvironmentsForLineno{equation}%
\patchBothAmsMathEnvironmentsForLineno{align}%
\patchBothAmsMathEnvironmentsForLineno{flalign}%
\patchBothAmsMathEnvironmentsForLineno{alignat}%
\patchBothAmsMathEnvironmentsForLineno{gather}%
\patchBothAmsMathEnvironmentsForLineno{multline}%
}

% Make this look more like JCP
\usepackage{newtxtext,newtxmath}
\usepackage[scaled=1.0]{helvet}
\usepackage[compact]{titlesec}
\usepackage{xcolor}
\usepackage{notes2bib}
\bibnotesetup{ note-name = , use-sort-key = false}
% Format section header
\titleformat{\section}
  {\normalfont\sffamily\bfseries}
  {\thesection.}{0.5 em}{\MakeTextUppercase}
\titlespacing{\section}{0pt}{12pt}{12pt}

\titleformat{\subsection}[block]
  {\normalfont\sffamily\bfseries}
  {\thesubsection.}{0.5 em}{}
\titlespacing{\subsection}{0pt}{12pt}{8pt}

\titleformat{\subsubsection}[block]
  {\normalfont\itshape\sffamily\bfseries\raggedright}
  {\arabic{subsubsection}.}{0.5 em}{}
\titlespacing{\subsubsection}{0pt}{8pt}{8pt}

\begin{document}

\title{Exploring Hilbert space on a budget II}
\author{Author, Authors}
\email{authors@calpoly.edu}
\affiliation{Department of Physics, California Polytechnic State University, San Luis Obispo, CA, 93410}

\begin{abstract}
This work explores the ability of quantum computational algorithms to represent the information content of Full Configuration Interaction (FCI) wave functions.

\end{abstract}

\linenumbersep=24pt

\maketitle

\section{\label{sec:intro}Introduction}

This is an introduction. It should include the following subtopics:

\begin{enumerate}
\item A review of the strong correlation problem (relevance, and difficulty for classical methods, discuss findings from EHS-1).
\item A discussion of the relevance of quantum algorithms to combat this problem, inherent circumvention of storage.
\item A discussion of the outstanding need for inter-comparison between the now wide variety of quantum algorithm classes that exist, resource estimation and accuracy, and comparison to classical methods. Investigate the quantum advantage boundary.
\item Also a need to compare the border of near-term and fault tolerant algorithms (not done often or at all).
\item Overview of previous quantum benchmark students, overview of the quantum algorithms we implemented, and classical algorithms from EHS-1,
\item Summary of findings (obviously later).
\end{enumerate}

This is a citation: \cite{Schriber2016Adaptive}

\section{\label{sec:theory}Theory}

\subsection{FCI wave functions and strong correlation}
\label{sec:fci_wf}
Given a basis of $K$ spin orbitals $\{\psi_p\}$ with $p = 1,\ldots,K$, we indicate a generic $N$-electron determinant $\ket{\psi_{i_1}\cdots\psi_{i_N}}$ using the notation $\ket{\Phi_I}$ where the multindex $I = (i_1,\ldots,i_N)$ represents an ordered list of indices ($i_1 < i_2 < \ldots < i_N$).
The set of $N$-electron determinants ($\mathcal{H}_N$) forms a Hilbert space of dimension $|\mathcal{H}_N| = \nfci$.
Using this notation, the FCI wave function is written as a linear combination of determinants, each parameterized by a coefficient ($C_{i_1,\ldots,i_N} \equiv C_{I} $)
\begin{equation}
\ket{\Psi_{\text{FCI}}} = \sum_{i_1 < i_2 < \ldots < i_N}^{K} C_{i_1 \cdots i_N} \ket{\Phi_{i_1 \cdots i_N}}  = \sum_{I}^{\nfci} C_{I} \ket{\Phi_{I}} 
\label{eq:fci_def1}.
\end{equation}
An equivalent way to express the FCI wave function employs occupation vectors.
In this representation, each determinant $\ket{\Phi_I}$ is associated with a vector of length $K$, $\ket{\mathbf{n}} =\ket{n_1, n_2, \ldots, n_K}$, where $n_i \in \{0,1\}$ is the occupation number of spin orbital $\psi_i$.
The FCI wave function represented in the occupation vector form is given by
\begin{equation}
\ket{\Psi_{\text{FCI}}}  = \sum_{\{  n_{i}\}} C_{n_{1} \ldots n_{K}} \ket{n_{1} \ldots n_{K}}
= \sum_{\mathbf{n}} C_{\mathbf{n}} \ket{\mathbf{n}}
\end{equation}
where the sum over all occupation vectors ($\{  n_{i}\} \equiv \mathbf{n}$) is restricted to $N$-electron determinants ($\sum_i n_j = N$) of given spin and spatial symmetry.

\subsection{Resources for Quantum Algorithms}
Overview of what contributes to runtime and accuracy.

\subsubsection{Classical Parameter Optimization}
Can probably keep some of this, will likely use here.
For a systematically improvable method $X$ we indicate the energy computed using $N_{\rm{par}}$  parameters as $E_X(N_{\rm{par}})$. We then define the accuracy volume, $\ncomp(\alpha)$, to be the smallest number of parameters such that the error per electron with respect to the FCI energy ($E_\mathrm{FCI}$) is less than or equal to $10^{-\alpha}$:
\begin{equation}
\label{eq:Vx}
\ncomp(\alpha) = N_{\rm{par}} : \frac{ |E_X(N_{\rm{par}}) - E_\mathrm{FCI}|}{N} \leq  10^{-\alpha}  \; E_\mathrm{h}.
\end{equation}

\subsubsection{Circuit Depth (NISQ and FT)}

\subsubsection{Measurement}

\subsubsection{Hamiltonian Factorization}


\subsection{Quantum Eigensolvers}
General ideal of optimized unitary, discuss discrepancy between PQE and VQE

Now list Ansatz

\subsubsection{dUCC}

particle hole dUCC, k-upG-UCCSD, AGP, symmetry adaptation.
Mention that some are introduced in layers

\subsubsection{Orbital Fabric}

\subsubsection{Tensor Network}

\subsubsection{Qubit CC}

\subsubsection{Hardware Efficient Ansatz}

\subsubsection{LDCA}

\subsubsection{Selected QE}

S-PQE, ADAPT, Iterative Qubit CC




\subsection{Quantum Imaginary Time Evolution}

\subsubsection{Fermionic 1st order QITE}

\subsubsection{Fermionic 1st order MVP QITE}



\subsection{Quantum Subspace Diagonalization}

\subsubsection{QSE}

\subsubsection{NO-VQE}

\subsubsection{MC-VQE}

\subsubsection{QLanczos}

\subsubsection{QKrylov}

\subsubsection{QDavidson}

\subsection{Quantum Phase Estimation}




\section{Hydrogen Model Systems}




\section{Computational Considerations}

\section{Results}

\section{Conclusions}

\bibliography{bibliography,extra}

\end{document}

























